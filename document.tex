%%This is a very basic article template.
%%There is just one section and two subsections.
\documentclass{article}
\usepackage[utf8]{inputenc}
\usepackage[ngerman]{babel}
\usepackage[T1]{fontenc}
\usepackage{ textcomp }
\begin{document}

{\Huge Rezepte}
\vspace{1cm}
\hrule
\vspace{1cm}
\tableofcontents
\newpage

\section{Pizzateig ohne Gehenlassen}
	\subsection*{Zutaten}
	\begin{itemize}
	  \item 500g Mehl
	  \item 1 Päckchen Trockenhefe
	  \item 200ml Milch
	  \item 80g Butter
	  \item 1 Ei
	  \item 1/2 Tl. Zucker
	  \item 1/2 Tl. Salz
	\end{itemize}
	Mehl mit Trockenhefe, Zucker und Salz vermischen.
	Butter in der Milch schmelzen und mit Mehl-Gemisch und Ei zu einem zähen Teig
	vermengen. Auf Blech ausbringen.\\
	Etwa 20-30 Minuten in den auf 250 Grad Celsius vorgeheizten Backofen.

\section{Elsässer Flammkuchen}
	\subsection*{Zutaten}
	\begin{itemize}
	  \item 200g Mehl
	  \item 2 El. Öl
	  \item 125ml Wasser
	  \item etwas Salz
	  \item 200g Cr\`{e}me double oder Cr\`{e}me fra\^{i}che
	  \item 100g Speck
	  \item 100g Zwiebeln
	\end{itemize}
	Aus Mehl, Öl, Wasser und Salz einen Knetteig herstellen.
	Mehl zugeben bis er nicht mehr klebrig ist.
	Zwiebeln in dünne, schmale Streifen schneiden.
	Den Teig sehr dünne ausrollen und mit Cr\`{e} me fra\^{i} che double
	bestreichen.
	Zwiebeln und Speck darauf verteilen.

	Im vorgeheizten Backofen auf höchster Stufe circa 15 bis 20 Minuten backen,
	bis der Boden knusprig ist.

\section{Cremiges Chili con Carne}
	\subsection*{Zutaten}
	\begin{itemize}
		\item 250 ml Sahne
		\item 250 g saure Sahne
		\item 2 Zehe/n Knoblauch
		\item 1 kl. Dose/n Tomatenmark
		\item 1 Paprikaschote(n), grün
		\item 480 g Kidneybohnen
		\item 300 g Mais
		\item 500 g Hackfleisch, nach Belieben
		\item gemischt
		\item 1 Würfel Brühe
		\item 1 große Zwiebel(n)
		\item 3 Tomate(n)
		\item 1 Chilischote(n), rot oder grün
		\item Margarine
		\item Salz und Pfeffer
	\end{itemize}
	Zwiebel, Knoblauch, sehr fein würfeln. Paprika
	ebenfalls in mundgerechte Stücke schneiden.
	Zwiebel in Margarine dünsten, bis die Würfel glasig
	sind, dann Paprika zugeben und etwa 3-4 Min.
	mitdünsten. Danach das Hack zugeben und anbraten,
	bis es braun wird. Große Hackstücke nachträglich mit
	einem Löffel klein stoßen. Ab und zu Margarine
	zugeben und etwas Wasser, aber erst, wenn das
	Hack schon leicht bräunlich ist. Wenn das Hack ganz
	braun ist, Tomatenmark, etwas Wasser und
	Knoblauch zugeben und ein wenig köcheln lassen.
	Die Tomaten abbrühen und häuten, Stiele entfernen
	und dann zerstoßen, bis ein Tomatenbrei entstanden
	ist, der ebenfalls zum Hack gegeben wird. Bohnen
	und Mais zugeben. Den Brühwürfel zerbröseln und in
	den Topf geben. Die Chilischote würfeln und zugeben
	oder ganz zum Chili geben, dann allerdings vor dem
	Servieren aus dem Topf holen und beiseite legen.
	Herd auf kleine Flamme stellen und Sahne zugeben,
	mit viel Pfeffer und Salz abschmecken. Noch 5 Min.
	köcheln lassen. \\
	Das Chili Con Carne auf Teller geben und dann auf
	jede Portion ein Klecks Sauerrahm!
	Anstelle des Sauerrahms kann man auch Schmand
	nehmen.

\section{NSFW Brownies}
\subsection*{Zutaten}
	\begin{itemize}
		\item 250 g Butter
		\item 100 g Vollmilchschokolade
		\item 100 g Zartbitterschokolade
		\item 125 g Walnüsse
		\item 80 g Kakaopulver
		\item 65 g Mehl
		\item 360 g Puderzucker
		\item 1 TL Backpulver
		\item 4 große Eier
	\end{itemize}
	Butter und Schokolade in einer Schüssel schmelzen.
	Walnüsse fein hacken und zu flüssiger Schokoladen-Buttermischung geben.
	In einer zweiten Schüssel Kakaopulver, Mehl, Puderzucker und Backpulver mischen.
	Schokoladen-Butter-Walnuss-Mischung hinzugeben und verrühren.
	Anschließend 4 Eier hinzugeben und abermals gut durchrühren.
	Den glatten Teig auf einem Backblech verteilen.\\
	Bei 180 Grad Celsius etwa 25 Minuten backen.

\section{Little Pots of Chocolate}
\subsection*{Zutaten}
	\begin{itemize}
		\item 250 ml Sahne
		\item 170 g Zartbitter Schokolade
		\item 2 Eier
		\item 2 EL Espresso, kalt
		\item 2 EL Likör (bspw. Kahlua)
	\end{itemize}
	Die Schokolade so fein wie möglich zerkleinern,
	die Sahne in einem Topf bis kurz vorm Kochen erhitzen und dann auf die Seite stellen.\\
	Schokolade, Eier, Espresso und Kahlua in einer Küchenmaschine mixen.
	Dann die heiße Sahne dazugeben und weiter verrühren,
	bis die Schokolade sich aufgelöst hat und eine gleichmäßige Farbe erreicht ist. \\
	Die Mischung auf 6 Espresso-Tassen aufteilen und im Kühlschrank für mindestens 2 Stunden kühl stellen.

\section{Italienischer Brotaufstrich}
\subsection*{Zutaten}
	\begin{itemize}
		\item 10 Basilikumblätter
		\item 1 Knoblauchzehe
		\item 5 getrocknete Tomaten, in Öl eingelegt
		\item 200 g Frischkäse
		\item 250 g Quark oder Schmand
		\item 1 gestrichener TL Kräutersalz
	\end{itemize}
	Basilikumblätter, Knoblauchzehe und getrocknete Tomaten klein hacken.
	Mit Frischkäse und Quark mischen und mit Kräutersalz abschmecken.
	Alternativ mit Pürierstab zubereiten.

\section{Pizzabrötchen}
\subsection*{Zutaten}
	 \begin{itemize}
		\item 600 g Mehl
		\item 500 g Quark
		\item 1 Päckchen Backpulver
		\item 1 TL Salz
		\item 2 EL Zucker
		\item 16 EL Milch
		\item 12 EL Öl
		\item 2x Cervelatwurst
		\item 2x Streukäse
		\item 100 g Röstzwiebeln
	\end{itemize}
	Teig aus Mehl, Backpulver, Quark, Milch, Öl, Salz und Zucker herstellen.
	Mit den restlichen Zutaten vermengen und in kleine Brötchen verteilen.
	Beim Umluft und 200 Grad Celsius für etwa 20 Minuten backen.

\section{Brüsseler Waffeln} % (fold)
\label{sec:bruesseler_waffeln}
\subsection*{Zutaten} % (fold)
\label{sub:zutaten}
	\begin{itemize}
		\item 120 g Zucker
		\item 200 g Butter
		\item 1/2 Päckchen Vanillezucker
		\item 2 Eier
		\item 1 TL Zitronenenschalen, abgerieben
		\item 350 g Mehl
		\item 1/2 Päckchen Backpulver
		\item 1 Prise Salz
		\item 150 ml Milch
		\item Wasser, lauwarm
	\end{itemize}
% subsection zutaten (end)
	Butter, Zucker, Zitronenschale und Vanillezucker schaumig rühren.
	Langsam die Eier hinzugeben. Mehl, Backpulver und Salz mischen und im Wechsel mit Milch
	unterrühren. Soviel Wasser hinzugeben bis ein flüssiger Teig endsteht.\\
	Teig 20 Minuten stehen lassen. Dann nochmals gut verrühren. Waffeleisen leicht fetten.

	Alternativ kann auch Creme fraiche und Milch verwendet werden.
% section bruesseler_waffeln (end)

\newpage
\section{Schokoladen Scones} % (fold)
\label{sec:schokoladen_scones}
\subsection*{Zutaten} % (fold)
\label{sub:zutaten}
	\begin{itemize}
		\item 225 g Mehl
		\item 2 TL Backpulver
		\item 60 g Butter, kalt
		\item 1 EL Zucker
		\item 50 g Schokoladendekor
		\item 150 ml Milch
	\end{itemize}
% subsection zutaten (end)

	Mehl mit Backpulver in eine Schüssel sieben.
	Butter in Flöckchen dazu geben und mit den Fingern zu einem krümeligen Teig verreiben.
	Zucker und Schokotropfen dazugeben und mit der Milch zu einem geschmeidigen Teig rühren,
	am besten mit einem Löffel.\\
	Den Teig auf einer leicht bemehlten Arbeitsfläche
	zu einem 12 cm * 12 cm Quadrat (circa 25 mm dick) ausrollen
	und in 9 Quadrate schneiden.
	Mit genügend Abstand auf ein leicht gefettetes Blech setzen,
	mit Milch bestreichen
	und bei 220 Grad Celsius (Ober/Unterhitze) im vorgeheizten Ofen circa 10-12 Minuten backen
	- bis sie goldgelb und schön aufgegangen sind.
% section schokoladen_scones (end)

\clearpage
\section{Apfeltiramisu} % (fold)
\label{sec:apfeltiramisu}
\subsection*{Zutaten} % (fold)
\label{sub:apfeltiramisu:zutaten}
	\begin{itemize}
		\item 250 g Speisequark
		\item 250 g Mascarpone
		\item 250 g Schlagsahne
    \item 1 Packung Löffelbiskuit
    \item 1 Glas Apfemus
    \item Apfesaft
    \item Calvados
    \item Zimt
    \item Zucker
    \item Vanillezucker
	\end{itemize}
% subsection zutaten (end)
  \subsection*{Zubereitung}
  \label{sub:apfeltiramisu:zubereitung}
  Quark und Mascarpone mischen. Sahne separat steif schalgen und
  unter die Quarkmasse heben. Mit Zucker und Vanillezucker nach
  belieben süßen.
  In einer Auflaufform eine Lage Löffelbiskuit,
  mit der Zuckerseite nach unten, auslegen.
  Apfelsaft mit Calvados im Verhältnis 3:1 mischen
  und den Löffelbikuit tränken.
  Nun abwechselnd Schichten Creme und Apfelmus auftragen,
  mit Creme abschließen.\\
  4-6 Stunden im Kühlschrank ziehen lassen,
  mit Zimt bestreuen und servieren.
% section  apfeltiramisu (end)

\clearpage
\section{Serviettenknödel} % (fold)
\label{sec:serviettenknoedel}
\subsection*{Zutaten} % (fold)
\label{sub:serviettenknoedel:zutaten}
	\begin{itemize}
		\item 8 helle Brötchen, vom Vortag
		\item 300 ml Milch, lauwarm
		\item 4 Eier
		\item etwas Muskatnuss, gerieben
		\item 1-2 EL Mehl
		\item 1 Prise Salz
		\item 1 EL Zwiebeln, fein gewürfelt
		\item \textonehalf Bund Petersilie, fein geschnitten
		\item 1-2 EL Butter
	\end{itemize}
% subsection zutaten (end)
  \subsection*{Zubereitung}
  \label{sub:serviettenknoedel:zubereitung}
	Brötchen würfeln und in einen Schüssel geben.
	Milch mit Eiern, Salz und Muskatnuss verrühren und
	Über die Brötchen geben.\\
	Zwiebel und Petersilie in heißer Butter anschwitzen und
	zu der Masse geben.
	Mehl darüber Streuen und
	alles zu einem geschmeidigen Teig vermischen.\\
	30 Minuten ziehn lassen.\\
	In einem weiten Topf reichlich Salzwasser zum Kochen bringen.
	Teig in zwei dicken Rollen formen und
	in jeweils einer angefeuchteten Serviete locker einwickeln und
	zubinden.\\
	Bei leicht siedendem Wasser etwa 30 Minuten garen.

\end{document}
