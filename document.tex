%%This is a very basic article template.
%%There is just one section and two subsections.
\documentclass{article}
\usepackage[utf8]{inputenc}
\usepackage[ngerman]{babel}
\begin{document}

{\Huge Rezepte}
\vspace{1cm}
\hrule
\vspace{1cm}
\tableofcontents
\newpage

\section{Pizzateig ohne Gehenlassen}
	\subsection*{Zutaten}
	\begin{itemize}
	  \item 500g Mehl
	  \item 1 Päckchen Trockenhefe
	  \item 200ml Milch
	  \item 80g Butter
	  \item 1 Ei
	  \item 1/2 Tl. Zucker
	  \item 1/2 Tl. Salz
	\end{itemize}
	Mehl mit Trockenhefe, Zucker und Salz vermischen.
	Butter in der Milch schmelzen und mit Mehl-Gemisch und Ei zu einem zähen Teig
	vermengen. Auf Blech ausbringen.\\
	Etwa 20-30 Minuten in den auf 250 Grad Celsius vorgeheizten Backofen. 

\section{Cremiges Chili con Carne}
	\subsection*{Zutaten}
	\begin{itemize}
		\item 250 ml Sahne
		\item 250 g saure Sahne
		\item 2 Zehe/n Knoblauch
		\item 1 kl. Dose/n Tomatenmark
		\item 1 Paprikaschote(n), grün
		\item 480 g Kidneybohnen
		\item 300 g Mais
		\item 500 g Hackfleisch, nach Belieben
		\item gemischt
		\item 1 Würfel Brühe
		\item 1 große Zwiebel(n)
		\item 3 Tomate(n)
		\item 1 Chilischote(n), rot oder grün
		\item Margarine
		\item Salz und Pfeffer
	\end{itemize}
	Zwiebel, Knoblauch, sehr fein würfeln. Paprika
	ebenfalls in mundgerechte Stücke schneiden.
	Zwiebel in Margarine dünsten, bis die Würfel glasig
	sind, dann Paprika zugeben und etwa 3-4 Min.
	mitdünsten. Danach das Hack zugeben und anbraten,
	bis es braun wird. Große Hackstücke nachträglich mit
	einem Löffel klein stoßen. Ab und zu Margarine
	zugeben und etwas Wasser, aber erst, wenn das
	Hack schon leicht bräunlich ist. Wenn das Hack ganz
	braun ist, Tomatenmark, etwas Wasser und
	Knoblauch zugeben und ein wenig köcheln lassen.
	Die Tomaten abbrühen und häuten, Stiele entfernen
	und dann zerstoßen, bis ein Tomatenbrei entstanden
	ist, der ebenfalls zum Hack gegeben wird. Bohnen
	und Mais zugeben. Den Brühwürfel zerbröseln und in
	den Topf geben. Die Chilischote würfeln und zugeben
	oder ganz zum Chili geben, dann allerdings vor dem
	Servieren aus dem Topf holen und beiseite legen.
	Herd auf kleine Flamme stellen und Sahne zugeben,
	mit viel Pfeffer und Salz abschmecken. Noch 5 Min.
	köcheln lassen. \\
	Das Chili Con Carne auf Teller geben und dann auf
	jede Portion ein Klecks Sauerrahm!
	Anstelle des Sauerrahms kann man auch Schmand
	nehmen.

\end{document}
